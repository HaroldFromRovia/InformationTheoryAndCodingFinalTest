\documentclass[../main.tex]{subfiles}
\setlength{\parskip}{0em}

\begin{document}

$x_1x_2x_3..x_{2n}=1$ тогда и только тогда, когда $x_1=x_2=..=x_{2n}=1$, а $\overline{x}_1\overline{x}_2..\overline{x}_{2n}=1$, тогда и только тогда, когда $x_1=x_2..=x_{2n}=0$. 

Таким образом, мы можем утверждать, что $x_1x_2x_3..x_{2n} \oplus x_1=x_2..=x_{2n} = 1$, тогда и только тогда, когда либо все аргументы равны 1, либо, когда все аргументы равны 0.

Рассмотрим $x_1\overline{x_2}x_3..\overline{x}_{2n}$. Данная функция равна 1 тогда и только тогда, когда $x_1=x_3=..=x_{2n-1}=1$, $x_2x_4x_{2n}=0$

Таким образом, наша характеристическая функция равна 1 в трех случаях. Когда Все аргументы равны 1, когда все аргументы равны 0, и когда все аргументы чередуются 1010..0. Но тогда очевидно, что метрика Хэмминга равна $n=d_{min}$ (сравнивать вариант со всеми нулями и всеми единицами нет смысла, так как есть вариант с меньшей метрикой). 

Значит код обнаруживает $d_{min}-1=n-1$ ошибок, а исправляет $\frac{d_{min}-1}{2}$
\end{document}