\documentclass[../main.tex]{subfiles}
\setlength{\parskip}{0em}

\begin{document}
   Пусть у нас есть некторое полное дерево. Пусть у дерева все листья равноудалены от корня, тогда неравенство Крафта принимает вид 
   $$\sum\limits_{i=1}^{M}s^{-l_i} = M*s^{-log_s(M)} = M*M^{-log_s(s)} = M^{1-log_s(s)} = 1 $$ 
   (помним свойство логарифмов $a^{log_b(c)}=c^{log_b(a)}$)
   
   Иначе говоря, для деревьев у которых все все листья равноудалены от корня неравенство Крафта переходит в равенство. Теперь докажем что это верно для любых полных деревьев.
   
   Пусть нам дано полное дерево, у которого не все листья равноудаленны от корня. Тогда добавим к 1 листу ещё 1 уровень, и посмотрим как изменится неравенство Крафта.
   
   Пусть изначально, значение $\sum\limits_{i=1}^{M}s^{-l_i}=N$, а расстояние до листа, к которому мы добавляем \newline уровень $l$. Тогда после добавления листьев мы получим $N-s^{-l} + s*s^{-l-1} = N$. То есть добавление $s$ ребер не изменило значение функции и таким образом мы можем дополнить наше дерево так, что все листья станут равноудалены от корня, а для данного случая, мы уже показали, что неравенство Крафта переходит в равенство, то есть $N=1$.
\end{document}
