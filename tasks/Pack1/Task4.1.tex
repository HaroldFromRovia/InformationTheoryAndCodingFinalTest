\documentclass[../main.tex]{subfiles}
\setlength{\parskip}{0em}

\begin{document}
Энтропия марковского стационарного источника представляет собой математическое ожидание условных энтропий системы относительно всех её состояний.

В задании была ошибка, преподаватель сказал заменить вероятность во 2-ой строке 1-го столбца на 0.

\begin{equation*}
P=
\left(
   \begin{array}{ccc}
     1/6 & 1/2 & 1/3 \\
     0   & 1/2 & 1/2 \\
     1/2 & 0   & 1/2
\end{array}
\right)
\end{equation*} 
$H(X) = \sum\limits_{i=0}^{n}p(x_i)*H(X|x_i)$ \\
$H(X|x_1) = \frac{2}{3} + \frac{1}{2}log(3)$ \\
$H(X|x_2) = 1$ \\
$H(X|x_3) = 1$ \\

Теперь найдем распределение стационарных вероятностей. Так как источник стационарен, то существует стохастический вектор удовлетворяющий уравнению $\overline{x}=\overline{x}P$, тогда для нахождения данного вектора, нам достаточно решить систему уравнений 
$$\overline{x}(P-E)=0$$ 

\begin{equation*} 
\left(
   \begin{array}{c}
     p(x_1)\\
     p(x_2)\\
     p(x_3)
\end{array}
\right) 
\left(
   \begin{array}{ccc}
     -5/6 & 1/2 & 1/3 \\
     0   & -1/2 & 1/2 \\
     1/2 & 0   & -1/2
\end{array}
\right) =0
\end{equation*}

Решая систему, получаем, что $p(x_1)=p(x_2)=p(x_3)$. Однако, существует дополнительное условие 
$\sum\limits_{x \in X}p(x) =1 \Rightarrow p(x_i)=\frac{1}{3}; i=\overline{(1,3)} \Rightarrow H(X) = \frac{1}{3}*(H(X|x_1)+H(X|x_2)+H(X|x_3)) = \frac{8}{9}+\frac{1}{6}log(3)$

\end{document}